\section{Introduction}

Text entry has been a field of research for the last twenty years, and innovations in text entry for mobile devices has seen many advances over the past few years. The importance of text entry is obvious -- technology is omnipresent with more smartphones than people. It is highly likely that most adults will use a smartphone to type something at least once per day. 

As such, the question of research into aiding typing behaviors is important as a matter of consequence. Autocorrect has been introduced to assist with disparities between a typed word and an intended word. Word suggestions have been added to the tops of virtual keyboards to help complete common phrases. Gesture typing, the action of dragging a finger continuously over letters to type words, has enjoyed great success. 

However, these innovations have been introduced as a roundabout way of deflecting from the real problem:  the standard QWERTY keyboard is not suited for use on mobile devices. Indeed, mobile devices often have to trade off screen space for typing accuracy, and the clustering of keys that the QWERTY keyboard employs does not make typing easier -- it makes it harder. Designed initially for typewriters, the layout persisted onto analog keyboards and was extended onto touchscreen devices as a legacy transition. However, the problems inherent in the design that weren't as big a problem on analog keyboards became magnified when typing on a small, constrained screen.

The accuracy of a person's ability to properly touch type a word -- and for the ability of a gesture recognizer to identify a swiped gesture as an intended word -- are confounded by the QWERTY layout. The QWERTY layout's close clustering of certain vowels and commonly used characters -- like u, i, and o -- inherently forces typing errors. This clustering makes it more difficult for error correction algorithms to correct user input~\cite{ Dunlop:2012:MPO:2207676.2208659}, and as the screen size shrinks the likelihood of errors also increases. For example, on the HTC Hero, the key centres are only 4.5mm apart~\cite{ Dunlop:2012:MPO:2207676.2208659}. 

As a result, researchers have devoted time and energy to develop new keyboards that can circumvent this. Multiple keyboards have been proposed that can achieve superior typing speeds for users, after some training time~\cite{ Dunlop:2012:MPO:2207676.2208659, lee2004top, Bi:2010:QSK:1753326.1753367, Bi:2016:IDO:2858036.285842}. The problem with these keyboards is the learning factor -- participants are unwilling to switch to keyboards because learning curves prevent them and increase typing errors in the short term.   The time it takes to learn a new keyboard is a key factor in a user's willingness to adopt a new keyboard design. Top-down learning techniques have been compared~\cite{lee2004top} to regular repetition on an ATOMIK keyboard layout, with results showing that it took half of the time for the top-down learning group to get comfortable with the keyboard layout than the control group.  The top-down learning group was also much more willing to use the keyboard in the future and had an overall more positive experience with it.  

To address this, some researchers have pivoted instead from designing the best keyboard by proposing the closest keyboard to QWERTY that allows for better typing speeds and reduced errors. To help create an easier transition into learning a new keyboard layout, Xiaojun Bi et al.~\cite{Bi:2010:QSK:1753326.1753367} created a keyboard conceptualized around the idea of finding a middle ground between familiar but inefficient (QWERTY), and unfamiliar but efficient --  the result of this idea is the Quasi-Qwerty keyboard~\cite{Bi:2010:QSK:1753326.1753367}.  This keyboard moves the keys on a QWERTY keyboard at most one key-space away from its original position in any direction.  The constraint of only moving the keys one space away allows the keyboard to keep some form of familiarity to the original QWERTY keyboard.

The problems with QWERTY have been well-documented, and yet it persists because of the learning curve. As a result, now researchers are focusing efforts on designing keyboards with an emphasis on similarities to QWERTY. This seems to be working towards invalidating the work done on keyboard optimization, where it is known that superior layouts exist but languish in obscurity~\cite{ Dunlop:2012:MPO:2207676.2208659}. We have better keyboards, but we can't use them. What can be done?

We propose a different take: if the problem proposed by researchers for adoption is the learning curve, can we make the learning easy enough for participants by breaking it into parts? Spaced repetition is a commonly used method for retaining information, where a person learns a fact or method by repeating it over time, each time increasing the interval between repeating the information. We hypothesize that the same can be done with a keyboard layout. We propose using the QWERTY keyboard as a starting point and transitioning to an optimized keyboard through a series of key replacements, allowing participants to practice the new keyboard over time. In this way, participants can learn small key switches at a time.

In this paper, we explore the effects of this kind of study design on participant's ability to learn keyboards.


  %Along with the Quasi-keyboard, they created a freely optimized keyboard in the study that does not include the Qwerty similarity constraint. They measured what they called the initial entry time.  The initial entry time is the time from when the word is displayed on the screen to when the user taps the last letter. The freely optimized keyboard showed to have the fastest initial entry times as well as the lowest error rate. Next was the Quasi-Qwerty keyboard. The predicted inputting speeds for Qwerty, Quasi-Qwerty, and the freely optimized keyboard were 181.2 characters per minute (cpm), 202.7 cpm, and 225 cpm respectively. Uncorrected error rates showed to be 2.9\% for Qwerty, 2.3\% for Quasi-Qwerty, and 1.4\% for the freely optimized layout. Although the freely optimized layout shows to be more optimal than the other two keyboards, it did take participants the longest amount of time to visually look for the keys.  

%Mark Dunlop and John Levine used Pareto optimization to design optimized keyboards for touch-screen devices [3].  Their study included Qwerty, Sath-trap, and Sath-rect.  Sath-trap is a trapezoidal shaped keyboard that was generated using the constraints of speed, familiarity, and spell correction. Sath-trap is the keyboard that best satisfied all of these constraints.  To add more improvement and familiarity to the iPhone keyboard, Sath-rect was created.  Sath-rect allows for the touch-screen keyboard to have larger keys therefore further improving the three constraints mentioned above. In this study the participants came to the lab for four days and were instructed to enter some phrases that were randomly selected from a set. The study showed that users reached 17.7 wpm by the fourth day using the Sath-Rect key while they were able t reach 21.3 wpm on Qwerty. Although participants were unable to reach their Qwerty speed on the new keyboard, Fits' law modeling confirmed a 10-11\% improvement in speed over the Qwerty keyboard.  The Sath-rect keyboard was compared to the Quasi-Qwerty keyboard and only showed a small improvement over Quasi-Qwerty.  
%
%In [3] the the study was conducted by having participants copy two initially presented phrases and then 17 more randomly selected phrases from a provided set which comes from [1]. There was a lot of time put in just for the participants to learn the optimized keyboards.  In total, the study took four days to complete and in those four days participants still were not able to reach the same speed as the get with the Qwerty Keyboard.  It is stated that they believe with more practice the users will be able to surpass the Qwerty keyboard results.  In the Quasi-Qwerty study [4] users were only told to type a specific word at a time out of a list of 19 possible words. These words were “the, and, you, that, is, in if, know, not, they, het, have, were, are, bit, quick, fox, jumps, and lazy”. The list of words was originally created in [5] where the authors describe the reasons for choosing this specific list of words. First of all, all of the letters in the alphabet are included in the list.  Second, the frequency of each letter in the list should be proportional their frequencies in the English language. Finally, the authors wanted the letter transitions to be representative of natural English. With that said, their study only lasted one day but was less involved in terms of how much the participants are typing.  



