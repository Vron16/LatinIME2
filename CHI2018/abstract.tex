Advocates for the adoption of keyboard designs that have been informed by research and insight have long been stymied by the persistence of the QWERTY layout in computing terminals.  Although text entry has improved on mobile devices, it is largely the result of several innovations independent of the actual keyboard layout, like autocorrect and word suggestion. The keyboard layout inherently creates problems by the clustering of vowels along a single line, particularly for gesture typing. The economics of QWERTY and the long lead time in learning a new keyboard creates difficulties for users to break free. In this paper, we conduct a study to see if touch keyboard users will find it easier to learn a new, optimized keyboard layout by using a procedural learning technique. Using this method, multiple keys on the keyboard will be switched at one time until the user adjusts to the placement of the new keys. Once the user shows signs of adjustment the next set of keys will be changed. Our results indicate ...